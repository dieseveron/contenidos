
\begin{questions}

\section{Administración de E/S}

\question Dispositivos
\begin{parts}
	\part Los dispositivos, según la forma de transferir los datos, se pueden clasificar en 2 tipos:
	\begin{itemize}
		\item Orientados a bloques
		\item Orientados a flujos
	\end{itemize}
	Describa las diferencias entre ambos tipos.
	\part Cite ejemplos de dispositivos de ambos tipos.
	\part Enuncie las diferencias que existen entre los dispositivos de E/S y que el SO debe considerar.
\end{parts}

\question Técnicas de E/S
Describa como trabajan las siguientes técnicas de E/S
\begin{itemize}
	\item E/S programada
	\item E/S dirigida por interrupciones
	\item DMA (Acceso Directo a Memoria)
\end{itemize}

\question La tecnica de E/S programa puede trabajar de dos formas:
\begin{itemize}
	\item E/S mapeada
	\item E/S aislada
\end{itemize}
Indique como trabajan estas 2 técnicas.

\question Enuncie las metas que debe perseguir un SO para la administración de la entrada salida.

\question Drivers
\begin{parts}
	\part ¿Qué son?
	\part ¿Qué funciones mínimas deben proveer?
	\part ¿Quién determina cuales deben ser estas funciones?
\end{parts}

\question Realice un grafico que marque la relación entre el Subsistema de E/S, los drivers, los controladores de dispositivos y los dispositivos.

\question Describa mediante un ejemplo los pasos mínimos que se suceden desde que un proceso genera un requerimiento de E/S hasta que el mismo llega al dispositivo.

\question Describa mediante un ejemplo los pasos mínimos que se suceden desde que un proceso genera un requerimiento de E/S hasta que el mismo llega al dispositivo.

\question Enuncie que servicios provee el SO para la administración de E/S.

\section{Administración de Discos}

\question Describa en forma sintética, cómo es la organización física de un disco, puede utilizar gráficos para mayor claridad.

\question La velocidad promedio para la obtención de datos de un disco esta dada por la suma de los siguientes tiempos:
\begin{itemize}
	\item \textit{Seek Time}
	\item \textit{Latency Time}
	\item \textit{Transfer Time}
\end{itemize}
De una definición para estos tres tiempos.

\question Suponga un disco con las siguientes características:
\begin{itemize}
	\item 7 platos con 2 caras utilizables cada uno.
	\item 1100 cilindros
	\item 300 sectores por pista, donde cada sector de es 512 \bytes.
	\item Seek Time de 10 \ms
	\item 9000 \rpm.
	\item Velocidad de Transferencia de 10 \mebishort\bytesshort/\s (\mebi\bytes \hspace{1pt} por segundos).
\end{itemize}
\begin{parts}
	\part Calcule la capacidad total del disco.
	\part ¿Cuantos sectores ocuparía un archivo de tamaño de 3 \mebishort\bytesshort (\mebi\bytes)?
	\part Calcule el tiempo de transferencia real de un archivo de 15 \mebishort\bytesshort (\mebi\bytes). grabado en el disco de manera secuencial (todos sus bloques almacenados de manera consecutiva)
	\part Calcule el tiempo de transferencia real de un archivo de 16 \mebishort\bytesshort (\mebi\bytes). grabado en el disco de manera aleatoria.
\end{parts}

\question El \textit{Seek Time} es el parámetro que posee mayor influencia en el tiempo real necesario para transferir datos desde o hacia un disco. Es importante que el SO planifique los diferentes requerimientos que al disco para minimizar el movimiento de la cabeza lecto-grabadora.

\hspace{20pt} Analicemos las diferentes políticas de planificación de requerimientos a disco con un ejemplo: Supongamos un \textit{Head} con movimiento en 200 \textit{tracks} (numerados de 0 a 199), que está en el track 83 atendiendo un requerimiento y anteriormente atendió un requerimiento en el track 75.

\hspace{20pt} Si la cola de requerimientos es: 86, 147, 91, 177, 94, 150, 102, 175, 130, 32, 120, 58, 66, 115. Realice los diagramas para calcular el total de movimientos de head para satisfacer estos requerimientos de acuerdo a los siguientes algoritmos de scheduling de discos:
\begin{parts}
	\part FCFS (\textit{First Come, First Served})
	\part SSTF (\textit{Shortest Seek Time First})	
	\part Scan	
	\part Look	
	\part C-Scan (\textit{Circular Scan})
	\part C-Look (\textit{Circular Look})
\end{parts}

\question ¿Alguno de los algoritmos analizados en el ejercicio anterior pueden causar inanición de requerimientos?

\question Supongamos un Head con movimiento en 300 pistas (numerados de 0 a 299), que esta en la
pista 143 atendiendo un requerimiento y anteriormente atendió un requerimiento en la pista 125.

\hspace{20pt} Si la cola de requerimientos es: 126, 147, 81, 277, 94, 150, 212, 175, 140, 225, 280, 50, 99, 118, 22, 55; y después de 30 movimientos se incorporan los requerimientos de las pistas 75, 115, 220 y 266. Realice los diagramas para calcular el total de movimientos de head para satisfacer estos requerimientos de acuerdo a los siguientes algoritmos de scheduling de discos:
\begin{parts}
	\part FCFS
	\part SSTF
	\part Scan	
	\part Look	
	\part C-Scan
	\part C-Look
\end{parts}

\question Supongamos un Head con movimiento en 300 pistas (numerados de 0 a 299), que esta en la pista 140 atendiendo un requerimiento y anteriormente atendió un requerimiento en la pista 135.

\hspace{20pt} Si la cola de requerimientos es: 99, 110, 42, 25, 186, 270, 50, 99, 147$^{PF}$, 81, 257, 94, 133, 212, 175, 130; y después de 30 movimientos se incorporan los requerimientos de las pistas 85, 15$^{PF}$, 202 y 288; y después de otros 40 movimientos más se incorporan los requerimientos de las pistas 75, 149$^{PF}$, 285, 201 y 59. Realice los diagramas para calcular el total de movimientos de head para satisfacer estos requerimientos de acuerdo a los siguientes algoritmos de scheduling de discos:
\begin{parts}
	\part FCFS
	\part SSTF
	\part C-Scan	
	\part Scan
	\part Look
	\part C-Look
\end{parts}

\section{Administración de Archivos}

\question Dados los siguientes métodos de administración de espació de un archivo:
\begin{itemize}
	\item Asignación contigua
	\item Asignación enlazada
	\item Asignación indexada
\end{itemize}
\begin{parts}
	\part Describa como trabaja cada uno.
	\part Cite ventajas y desventajas de cada uno.
\end{parts}

\question Gestión de espacio libre.  Dados los siguientes métodos de gestión de espacio libre en un disco:
\begin{itemize}
	\item Tabla de bits
	\item Lista Ligada
	\item Agrupamiento
	\item Recuento
\end{itemize}
\begin{parts}
	\part Describa como trabajan estos métodos.
	\part Cite ventajas y desventajas de cada uno.
\end{parts}

\question Gestión de archivos en \textit{UNIX}. 

\hspace{20pt} El sistema de archivos de UNIX utiliza una versión modificada del esquema de Asignación Indexada para la administración de espacio de los archivos.

\hspace{20pt} Cada archivo o directorio esta representado por una estructura que mantiene, entre otra información, las direcciones de lo bloques que contienen los datos del archivo: el I-NODO.

\hspace{20pt}  Cada I-NODO contiene 13 direcciones a los bloques de datos, organizadas de la siguiente
manera:
\begin{itemize}
	\item 10 de direccionamiento directo.
	\item 1 de direccionamiento indirecto simple.
	\item 1 de direccionamiento indirecto doble.
	\item 1 de direccionamiento indirecto triple.
\end{itemize}
\begin{parts}
	\part Realice un grafico que describa la estructura del \textit{I-NODO} y de los bloques de datos. Cada bloque es de 1 \kibishort\bitsshort (\kibi\bits). Si cada dirección para referenciar un bloque es de 32 bits:
	\begin{subparts}
		\subpart ¿Cuántas referencias (direcciones) a bloque pueden contener un bloque de disco?
		\subpart ¿Cuál sería el tamaño máximo de un archivo?
	\end{subparts}
\end{parts}

\question Analice las siguientes fórmulas necesarias para localizar un I-NODO en la lista de inodos:

\fbox{\begin{minipage}{40em}
\textbf{nro bloque} = ((nro de inodo -1)/nro. de inodos por bloque) + bloque de comienzo de la lista de inodos.
\end{minipage}}

\fbox{\begin{minipage}{40em}
\textbf{Desplazamiento del inodo en el bloque} = ((nro de inodo - 1) módulo (número de inodos por bloque)) * medida de inodo del disco.
\end{minipage}}

\begin{parts}
	\part Según la primer fórmula, asumiendo que en el bloque 2 está en el comienzo de la lista de inodos y que hay 8 inodos por bloque: calcule donde se encuentra el inodo 8 y el 9. ¿Dónde estarían para bloque de disco de 16 inodos?
	\part De acuerdo a la segunda fórmula, si cada inodo del disco ocupa 64 \bytes y hay 8 inodos por bloque de disco, el inodo 8 comienza en el desplazamiento 448 del bloque de disco. ¿Dónde empieza el 6? Si fueran inodos de 128 \bytes y 24 inodos por bloque: ¿dónde empezaría el inodo 8?
\end{parts}

\end{questions}