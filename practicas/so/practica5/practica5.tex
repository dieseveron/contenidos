\section{Requisisitos}
\textit{Para poder realizar la siguiente práctica es necesario tener instalado un sistema operativo Linux con un 
espacio libre de 2GiB como mínimo. Además, se deberán instalar los paquetes necesarios para poder generar el RAID 
y LVM.}\\
Obs. 1: al final de la práctica se encuetran algunos links de referencia para las siguientes preguntas\\
Obs. 2: en el pdf de la explicación de la práctica, al final de la misma, se encuentran algunas imágenes de como
realizar un particionado manual. No es necesario seguirlo. Lo importante es que al momento de la instalación quede
en el disco un espacio libre de 2GB para poder crear las nuevas particiones que se usarán para realizar la
práctica.

\section{Conceptos Generales}
\begin{questions}
  \question ¿Qué es un file system? 

  \question Describa las principales diferencias y similitudes entre los file systems: FAT, NTFS, Ext(2,3,4), XFS, HFS+ y ZFS

  \question ¿Qué función cumple el directorio lost+found en ext3? 

  \question ¿Qué es el particionado? ¿Qué es el UUID? ¿Para qué se lo utiliza? (Hint: ver el comando blkid)
  
  \question ¿Qué es el área de swap en Linux? ¿Existe un área similar en Windows?

  \question En Linux, ¿dónde se almacena el nombre y los metadatos de los archivos?

  \question Seleccione uno de sus file systems (una partición) y conteste usando el comando dumpe2fs:
  \begin{itemize}
   \item ¿Qué información describe el comando dumpe2fs?
   \item ¿Cuál es el tamaño de bloque del file system?
   \item ¿Cuántos inodos en total contiene el file system? ¿Cuántos archivos como máximo se pueden crear
    con el estado actual del file system? 
   \item ¿Cuántos grupos de bloques existen? 
  \end{itemize}

  \question ¿Qué es el directorio /proc? Usando este directorio, contestar: 
   \begin{itemize}
    \item ¿Cuál es la versión de SO que tiene instalado?
    \item ¿Cuál es procesador de su máquina? 
    \item ¿Cuánta memoria RAM disponible tiene? 
   \end{itemize}
 
  \question Usando el comando stat, contestar:
    \begin{itemize}
     \item ¿Cuándo fue la última vez que se modificó el archivo /etc/group? 
     \item ¿Cuál es la diferencia entre los datos Cambio (Change) y Modificación (Modify)? 
     \item ¿Cuál es el inodo que ocupa? 
     \item ¿Cuántos bloques ocupa?
    \end{itemize}

  \question ¿Qué es un link simbólico? ¿En qué se diferencia de un hard-link?

  \question Si tengo un archivo llamado prueba.txt y le genero un link simbólico, ¿qué sucede con el link simbólico si
    elimino el archivo prueba.txt? ¿Y si el link fuese hard-link? (Hint: probar usando el comando ln)

  \question ¿Para qué sirven los permisos especiales en Linux? Analizar el sticky-bit, setuid, setgid
 
  \question ¿Cuáles son los permisos del archivo /etc/passwd? ¿Por qué puedo modificar mi password si no soy usuario root?
    (Hint.: observe los permisos especiales del archivo ejecutable passwd)

  \question Crear un archivo en el directorio /tmp. Si abre otra consola y se loguea con un usuario distinto, ¿puede borrar 
    ese archivo? ¿Por qué? (Hint.: ver permisos especiales en Linux)

  \question ¿Qué es un RAID? Explique las diferencias entre los distintos niveles de RAID 

  \question ¿Qué es LVM? ¿Qué ventajas presenta sobre el particionado tradicional de Linux? 

  \question ¿Cómo funcionan los ``snapshots'' en LVM? 
\end{questions}

\section{RAID}

En este punto se va a crear un RAID Level 5 por software. Para simular los 3 discos necesarios para un RAID de este nivel se deberán
generar 3 particiones del mismo tamaño. Una vez realizado este paso se generará el RAID. Para esto se utilizará la herramienta mdadm 
que deberá descargarla con las herramientas apt/yum.

\begin{enumerate}
 \item  Utilizar el comando parted -l (debe ejecutarlo con sudo) para ver la tabla de particiones. Conteste:
  \begin{enumerate}  
   \item ¿Cómo llama Linux al dispositivo físico? ¿Cuál es su tamaño total?
   \item ¿Cuántas particiones existen? ¿Qué tipo de file-system contiene cada una?
   \item ¿Qué significa la bandera arranque?
  \end{enumerate}
 \item Usando el comando parted crear una nueva partición de tipo extendida (debe seleccionar el dispositivo donde se
    van a generar las particiones):
  
    {\it\small 1) sudo parted}\\
    {\it\small 2) (parted) print}\\ 
    {\it\small 3) (parted) mkpart}\\
    {\it\small 4) Tipo de partición: extendida}\\
    {\it\small 5) Inicio: \#MB} {\tiny (igual al tamaño de la partición existente más alto}\\
    {\it\small 6) Fin: \#MB} {\tiny (igual al tamaño máximo del dispositivo físico)}\\

    Dentro de esta partición extendida crear 3 nuevas particiones de 300MB cada una. Para esto utilizar nuevamente 
    el comando mkpart pero debe seleccionar logical como tipo de partición. Como tipo de sistema de archivos elija ext3.\\
    Obs.: como las particiones lógicas se crean dentro de una partición extendida, los valores de inicio/fin de 
     cada partición lógica deben estar dentro de los valores seleccionados en la partición extendida. 
 
 \item ¿Por qué es necesario crear una partición extendida?  
 \item Mirar nuevamente la tabla de particiones para ver las nuevas particiones y salga del entorno parted
 \item  Utilizar el comando mdadm para crear un RAID 5 utilizando las 3 particiones lógicas que se generaron en el punto anterior:
    \begin{verbatim}
      mdadm –create --verbose /dev/md0 --level5 --raid-devices=3 /dev/sda5 
                /dev/sda6 /dev/sda7
    \end{verbatim}
   (Obs.: md0 es el nombre que le dará al nuevo RAID)
 \item ¿Qué significan los valores sda5, sda6 y sda7? 
 \item Ejecutar la siguiente consulta y contestar:
   \begin{verbatim}
       mdadm --detail /dev/md0
   \end{verbatim}
 \begin{enumerate}
  \item ¿Cuál es el tamaño total del RAID?
  \item ¿Cuál es el tamaño total utilizable para almacenar datos?
 \end{enumerate}
 \item Analizar el contenido del siguiente comando:
   \begin{verbatim}
       cat /proc/mdstat
   \end{verbatim} 
 Obs.: puede suceder que al reiniciar la VM el RAID se vea como de solo lectura y con el número 127. Para solucionar esto deben
  ejecutar los comandos {\it mdadm --stop /dev/md127}, para parar el RAID, y {\it mdadm --assemble --scan} para volverlo a generar
  como md0 y de lectura/escritura. Esto se debe hacer cada vez que se inicia la VM. Si quiere que quede en forma persistente a 
  través de los reboots debe guardar la configuración en el archivo mdadm.conf, {\it mdadm --assemble --scan $>>$ /etc/mdadm/mdadm.conf}
  y luego {\it update-initramfs -u} (esto último puede tardar un poco de tiempo)

 \item Ahora se va a probar la funcionalidad del RAID 5. Para esto completar los siguientes pasos:
 \begin{enumerate}
  \item Crear un file system de tipo ext3 en el RAID 5 recién generado
    \begin{verbatim}
       mkfs.ext3 /dev/md0
    \end{verbatim}
  \item Montar la partición con el file system generado en el directorio /mnt/rd5
  \item Crear un directorio con dos archivos 
  \item Quitar una de las particiones del RAID. Para esto ponemos uno de los componentes en falla:
    \begin{verbatim}
       mdadm --fail /dev/md0 /dev/sda7 
    \end{verbatim}
  \item Observar el estado del RAID y contestar
  \begin{enumerate} 
   \item ¿Cuál es el estado del RAID? ¿Cuántos dispositivos activos existen?
   \item ¿Qué sucedería si se ejecuta el comando anterior sobre una de las particiones restantes?
  \end{enumerate}
  \item Quitar del RAID el componente puesto en falla en el paso anterior
    \begin{verbatim}
       mdadm -r /dev/md0 /dev/sda7
    \end{verbatim}
  \item Observar nuevamente el estado del RAID y contestar:
  \begin{enumerate}
   \item ¿Se puede acceder al directorio /mnt/rd5? ¿Están los archivos creados anteriormente?
   \item ¿Qué hubiese sucedido si teníamos otra partición como ``hot-spare''?
  \end{enumerate}
  \item Por último, remover la partición permanentemente del RAID (Obs.: esto es muy importante para que el próximo 
    booteo mdadm no intente usar a esta partición como parte del RAID, lo que provocaría la pérdida de todos los datos)	 
    \begin{verbatim} 
       mdadm --zero-superblock /dev/sda7
    \end{verbatim}
    A partir de este momento la partición /dev/sda7 se puede utilizar como una partición común	 		 	
 \end{enumerate}
 \item Para evitar la pérdida de datos es fundamental volver al RAID a un estado estable (sacarlo del estado degradado). 
    Para esto se agregará nuevamente la partición /dev/sda7 que se quitó en el paso anterior
  \begin{enumerate}
   \item Ejecutar el comando mdadm --add /dev/md0 /dev/sda7
   \item Ejecutar el comando mdadm –detail /dev/md0
   \item ¿Qué hace el RAID con la nueva partición recientemente agregada? ¿Qué significa el estado ``Rebuild Status''?
   \item ¿Es posible ingresar al recurso /mnt/rd5? ¿Se encuentran disponibles los datos creados en el punto anterior?
  \end{enumerate}
  
 \item Como los datos que mantiene el RAID son muy importantes es neceario tener un disco (partición en nuestro ejemplo) 
   de respaldo. Para esto se va a agregar una partición como ``hot-spare''. 
 \item Usando el comando parted generar una nueva partición, /dev/sda8
 \item Agregar la nueva partición al RAID:
   \begin{verbatim}
       mdadm --add /dev/md0 /dev/sda8
   \end{verbatim}
   \begin{enumerate}
     \item ¿Cómo se agregó la nueva partición? ¿Por qué?
   \end{enumerate}           
 \item Volver a poner en falla a la partición /dev/sda7. Ver el estado del RAID y contestar   
   \begin{verbatim}
       mdadm --detail /dev/mda0 
   \end{verbatim}
  \begin{enumerate}
   \item ¿Qué hace el RAID con la partición que estaba como spare?
  \end{enumerate} 

 \item Por último, se eliminará el RAID creado en los pasos anteriores:
  \begin{enumerate}
   \item Desmontar el RAID (comando umount) 
   \item Para cada una de las particiones del RAID ejecutar los pasos realizados cuando se quitó una partición del RAID 
     (mdadm con las opciones -–fail y -r). Por cada partición que se quita ir mirando el estado del RAID para ver como se comporta         
   \item Remover los superbloques de cada una de las particiones
    \begin{verbatim} 
       mdadm --zero-superblock /dev/sda5 /dev/sda6 /dev/sda7
    \end{verbatim}
   \item Remover el RAID
    \begin{verbatim} 
       mdadm --remove /dev/md0 
    \end{verbatim}
   Obs.: si existe, quitar la ĺinea ARRAY... del archivo /etc/mdadm/mdadm.conf
   \item Reiniciar y comprobar que el RAID ya no existe
  \end{enumerate}
\end{enumerate}
     
\section{LVM (Logical Volumen Management}

\textit{A continuación se creará un LVM utilizando las particiones /dev/sda5, /dev/sda6 y /dev/sda7 (respetar el tamaño y nombre 
de los volúmenes y directorios). En principio solo se utilizarán las particiones 5 y 6, luego se agregará la partición 7 
para incrementar el tamaño de los volúmenes}
 
\begin{enumerate}
 \item Instalar la herramiento lvm2 con apt/yum
 \item Indicar que los particiones 5 y 6 funcionarán como volúmenes físicos
  \begin{verbatim}
    pvcreate /dev/sda5 /dev/sda6
  \end{verbatim}
 \item Mediante el comando pvdisplay observar el estado del volumen físico recientemente creado
 \item Crear un grupo de volúmenes (volume group, VG) llamado “so\_vg”
  \begin{verbatim}
    vgcreate so\_vg /dev/sda5 /dev/sd6
  \end{verbatim}
 \item Utilizar el comando vgdisplay para ver el estado del VG
 \begin{itemize} 
  \item ¿Cuál es tamaño total del VG?
  \item ¿Qué significa PE?
 \end{itemize}
 \item Crear dos volúmenes lógicos (logical volume, LV) de 8MB y 120MB respectivamente
  \begin{verbatim}
       lvcreate -l 2 -n lv\_vol1 so\_vg
       lvcreate -L 120M -n lv\_vol2 so\_vg
  \end{verbatim}
 \item ¿Cuál es la diferencia entre los dos comandos utilizados en el punto anterior?
 \item Formatear los dos LV generados en el paso anterior con un file system de tipo ext3:
   \begin{verbatim}
       mkfs.ext3 /dev/so\_vg/lv\_vol1
   \end{verbatim}
 \item Crear dos directorios, vol1 y vol2, dentro de /mnt y montar ambos LG en estos directorios (montar el LV lv\_vol1 en el 
   directorio vol1 y lv\_vol2 en el directorio vol2)
 \item Ejecutar el comando {\it proof} (Puede tomar un rato su ejecución. Este comando estará disponible en la plataforma y deberán
   copiarlo a la VM)
 \item Crear un nuevo archivo en /mnt/vol1. ¿Es posible? ¿Por qué?
 \item Es posible solucionar esto incrementando el tamaño del file system. Para esto primero se debe incrementar el tamaño del LV
   correspondiente:
 \begin{enumerate}
  \item Desmontar el LV lv\_vol1
  \item Extender el LV lv\_vol1 en 20M
    \begin{verbatim} 
       lvextend -L +20M /dev/so\_vg/lv\_vol1
    \end{verbatim}
  \item Incrementar el tamaño del file system. Antes de incrementar el tamaño es necesario ejecutar el comando e2fsck
    para comprobar el file system esté correcto:
     \begin{verbatim}
       e2fsck -f /dev/so\_vg/lv\_vol1 
       resize2fs /dev/so\_vg/lv\_vol1	
     \end{verbatim}
 \end{enumerate}
 \item Montar nuevamente el LV en /mnt/vol1
 \item ¿Siguen estando los datos disponibles?
 \item Intentar crear un nuevo archivo en /mnt/vol1. ¿Es posible? ¿Por qué?
 \item Se desea crear un nuevo LV de 500M. ¿Hay suficiente espacio? 
 \item Para solucionar el punto anterior se debe extender el tamaño del VG, por lo tanto, se agregará la partición sda7 al VG:
   \begin{verbatim}
     pvcreate /dev/sda7
     vgextend so\_vg /dev/sda7
   \end{verbatim}
 \item Comprobar con los comando correspondientes que se haya extendido el tamaño del VG
 \item Generar el nuevo LV de 500M (llamarlo lv\_vol3)
 \item Montar este nuevo LV en el directorio /mnt/lv\_vol3 y crear un directorio con dos archivos adentros (los nombre pueden ser cualquiera)
 
 En el siguiente paso se intentará reducir el tamaño del LV generado en el punto 18) de 500M a 400M

 \item Desmontar el LV.  
 \item Reducir el LV lv\_vol3 (decir Sí/Yes al aviso que aparece al ejecutar el siguiente comando)
  \begin{verbatim}
    lvreduce -L 400M /dev/mapper/so\_vg-lv\_vol3
  \end{verbatim}
 \item Montar el LV reducido en el paso anterior. ¿Es posible ver el directorio y los archivos generados en el paso anterior?
 \item ¿Cuál es el error en el procedimiento anterior? ¿Cuáles serían los pasos correctos? 
 \item Ejecute los pasos necesarios y de forma ordenada para poder achicar el LV a 400M (investigar)
\end{enumerate}

\vspace{0.3cm}

\begin{itemize}
\item http://www.nongnu.org/ext2-doc/ext2.html
\item http://e2fsprogs.sourceforge.net/ext2intro.html
\item http://www.cyberciti.biz/tips/understanding-unixlinux-filesystem-superblock.html
\item http://www.tldp.org/LDP/tlk/fs/filesystem.html
\item http://www.geekride.com/hard-link-vs-soft-link/
\end{itemize}


\end{document}

