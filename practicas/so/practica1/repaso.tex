\section{Repaso general}

\textit{El objetivo de esta primera parte de la práctica es repasar los
  conceptos de \textbf{shell scripting} aprendidos en la materia
  \textbf{Introducción a los Sistemas Operativos}. Se realizará un repaso
  general sobre comandos los comandos más comunes y su uso en
  \textbf{scripts}}.

\subsubsection{\textit{Scripts}}

\begin{questions}
\question Relice un \textit{script} que guarde en el archivo \texttt{/tmp/usuarios} los
nombres de los usuarios del sistema cuyo \textit{UID} sea mayor a 1000.
\question Implemente un \textit{script} que reciba como parámetro el nombre de un
proceso e informe cada 15 segundos cuántas instancias de ese proceso están en
ejecución.
\question Desarrolle un \textit{script} que guarde en un arreglo todos los archivos del
directorio actual (incluyendo sus subdirectorios) para los cuales el usuario que ejecuta
el \textit{script} tiene permisos de \textbf{ejecución}. Luego, implemente las siguientes
funciones:
\begin{parts}
  \part \texttt{cantidad}: Imprime la cantidad de archivos que se encontraron
  \part \texttt{archivos}: Imprime los nombres de los archivos encontrados en orden alfabético
\end{parts}
\question Se le ha encomendado organizar las fotos (en formato \texttt{jpg}) de todos los eventos
de los que su empresa ha participado en el último año, los cuales se encuentran organizados en
directorios con el nombre del evento. Para facilitar su búsqueda posterior, los archivos deben
tener nombres que sigan el siguiente patrón: \texttt{EVENTO-N.jpg}, donde:
\begin{itemize}
  \item \texttt{EVENTO} es el nombre del evento (el del directorio que se está procesando)
  \item \texttt{N} es un índice de foto, comenzando en \texttt{1}
\end{itemize}

Realice un \textit{script} que renombre los archivos de cada subdirectorio del directorio actual
siguiendo lo especificado en el párrafo anterior.

\textbf{Ejemplo:} dada la siguiente estructura de archivos y directorios:

\begin{lstlisting}

bashconf-15/
  DSC01050.jpg
  DSC01051.jpg
  DSC01052.jpg
  DSC01053.jpg
  DSC01054.jpg
jsconf-14/
  DSC01230.jpg
  DSC01231.jpg
  DSC01232.jpg
  DSC01235.jpg
  DSC01236.jpg
oktoberfest-14/
  DSC02229.jpg
  DSC02230.jpg
  DSC02231.jpg
  DSC02232.jpg

\end{lstlisting}

Se desea terminar con la siguiente estructura luego de ejecutar su \textit{script}:

\begin{lstlisting}

bashconf-15/
  bashconf-15-1.jpg
  bashconf-15-2.jpg
  bashconf-15-3.jpg
  bashconf-15-4.jpg
  bashconf-15-5.jpg
jsconf-14/
  jsconf-14-1.jpg
  jsconf-14-2.jpg
  jsconf-14-3.jpg
  jsconf-14-4.jpg
  jsconf-14-5.jpg
oktoberfest-14/
  oktoberfest-14-1.jpg
  oktoberfest-14-2.jpg
  oktoberfest-14-3.jpg
  oktoberfest-14-4.jpg

\end{lstlisting}

\end{questions}

\subsubsection{Preguntas de repaso}

\begin{questions}
\question ¿Qué es la shell? ¿Para qué sirve?
\question ¿En qué espacio (usuario/kernel) la ubicaría?

\question Si pensamos en el funcionamiento de una shell básica podríamos
detallarlos secuencialmente de la siguiente manera:
\begin{itemize}
  \item Esperar a que el usuario ingrese un comando
  \item Parsear lo que el usuario tipeo para obtener el comando con sus respectivos argumentos
  \item Crear un nuevo proceso para ejecutar el comando ingresado por el usuario
  \item En el nuevo proceso ejecutar el comando y retornar con el estado de dicha ejecución
  \item En el proceso padre, esperar a que el proceso hijo termine
  \item Volver a empezar.
\end{itemize}

\question Investigue las system call \textit{fork}:
\begin{parts}
  \part Que es lo que realiza?
  \part ¿Que retorna?
  \part ¿Para que podrian servir los valores que retorna?
  \part ¿Por que invocaria a la misma al implementar una shell?
\end{parts}

\question Investigue la system call \textit{exec}:
\begin{parts}
  \part ¿Para qué sirve?
  \part ¿Comó se comporta?
  \part ¿Cuáles son sus diferentes declaraciones POSIX?
\end{parts}

\question Investigue la system call \textit{wait}:
\begin{parts}
\part ¿Para qué sirve?
\part Sin ella, ¿qué sucedería, pensando en la implementación de la shell?
\end{parts}

\end{questions}
