\section{Redes y Sistemas Operativos}

\textit{En esta sección se aprenderá cómo integrar conceptos del sistema
  operativo, como redirecciones y \textbf{pipes}, con una red TCP/IP. Para
  ello se utilizará principalmente la herramienta \texttt{netcat}, también
  conocida como \texttt{nc}. Investigue
  su funcionamiento y resuelva. \textbf{Tip:} \texttt{man nc}}

\begin{questions}
  \question La herramienta \texttt{netcat} provee una forma sencilla de
  establecer una conexión TCP/IP. En una terminal levante una sesión de
  \texttt{netcat} en modo servidor, que escuche en la IP 127.0.0.1
  (localhost) en un puerto a elección. En otra terminal conéctese, también
  vía \texttt{netcat}, al servidor recién levantado. Interactúe y
  experimente con ambas terminales.

  \question \texttt{netcat} también es bueno al momento de transmitir
  archivos sobre una red TCP/IP. Utilizando dos terminales como se hizo en
  el ejercicio anterior, transmita el archivo \texttt{/etc/passwd} desde
  una sesión de \texttt{netcat} hacia la otra.

  \textbf{Tip:} recordar \textit{pipes} y redirecciones.

  \question Desarrolle un \textit{script} que reciba como parámetro una
  lista de \textit{hosts} con el puerto 80 abierto y, para cada uno,
  realice un requerimiento HTTP GET a la URI raíz y devuelva el valor del
  campo \texttt{Content-Length} de la respuesta.
  \begin{lstlisting}
./so-contentlength.sh www.google.com www.debian.org www.linti.unlp.edu.ar
www.google.com: 262
www.debian.org: 470
www.linti.unlp.edu.ar: 223
  \end{lstlisting}

  \textbf{Tip:} \texttt{printf "GET / HTTP/1.0\textbackslash
    r\textbackslash n\textbackslash r\textbackslash n" | ... }

  \question Interprete y describa qué es lo que hace el siguiente fragmento
  de código extraído de la \textit{man page} de \texttt{netcat}.
  \begin{lstlisting}

$ rm -f /tmp/f; mkfifo /tmp/f
$ cat /tmp/f | /bin/sh -i 2>&1 | nc -l 127.0.0.1 1234 > /tmp/f
  \end{lstlisting}

  \textbf{Tip:} \texttt{man mkfifo}

  \question Desarrolle un \textit{script} que permita al usuario chatear
  con otra instancia del \textbf{mismo} \textit{script}. Para ello, el
  \textit{script} deberá recibir como parámetro si va a funcionar como
  \textbf{(c)}liente o como \textbf{(s)}ervidor. También deberá recibir
  como parámetro un \textit{nickname} para el usuario. Por ejemplo, para
  invocar el \textit{script} en modo servidor con el \textit{nick}
  \textbf{jvg}, debería ejecutar:

  \begin{lstlisting}

 ./nc-chat.sh s jvg
  \end{lstlisting}

Además, los mensajes transmitidos deben comenzar con la fecha, hora y
\textit{nick} del emisor. Por ejemplo:
\begin{lstlisting}
  
Thu Mar 12 13:03:14 ART 2015, jvg says:
Knock knock Neo.
\end{lstlisting}  

\end{questions}
