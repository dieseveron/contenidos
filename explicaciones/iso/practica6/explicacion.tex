

%%%%%%%%%%%%%%%%%%%%%%%%%%%%%%%%%%%%%%%%%%%%%%%%%%%%%%%%%%%%%%%%%%

\begin{comment}

\begin{frame}[fragile]
  \frametitle{Características - Configuración de discos (cont.)}
  \begin{itemize}
	  \item A futuro, todos los dispositivos llamados hdX serán denominados sdX $\leftarrow$ Introducido en Debian/Squeeze
	  \item Por estas y otras razones se adoptan 4 mecanismos nuevos para nomenclar\footnote{\url{http://wiki.debian.org/Part-UUID}}:
	  \begin{itemize}
	  	\item Nombres persistentes por \textbf{UUID} (\small{Universal Unique Identifier}):
	  	\begin{lstlisting}
$ ls –l /dev/disk/by-uuid/
2d781b26-0285-421a-b9d0-d4a0d3b55680 -> ../../sda1
31f8eb0d-612b-4805-835e-0e6d8b8c5591 -> ../../sda7
		\end{lstlisting}
		\item Utilizando \textbf{labels}
		\begin{lstlisting}
$ ls -l /dev/disk/by-label
data -> ../../sdb2
data2 -> ../../sda2
		\end{lstlisting}
	  \end{itemize}
  \end{itemize}
\end{frame}

\begin{frame}
  \frametitle{Herramientas para particionar}
  \begin{itemize}
	  \item El particionado de un disco se lo puede realizar mediante:
	  \begin{itemize}
	  	\item Software destructivo: \textit{fdisk}
	  	\item Software no destructivo: \textit{fips}, \textit{gparted}
		\begin{figure}
		    \includegraphics[scale=0.3]{images/gparted.png}
		\end{figure}
	  \end{itemize}
  \end{itemize}
\end{frame}

\begin{frame}[fragile]
  \frametitle{Permisos}
  \begin{itemize}
	  	\item Se aplican a directorios y archivos
	  	\item Existen 3 tipos de permisos y se basan en una notación octal:
	  	\begin{table}
		      \centering
		      \resizebox{10pc}{!}{
			  \begin{tabular}{| c | c | c |}
			      \hline
			      \bf Permiso & \bf Valor & \bf Octal \\
			      \hline
			      Lectura & R & 4 \\
			      \hline
			      Escritura & W & 2 \\
			      \hline
			      Ejecución & X & 1 \\
			      \hline
			  \end{tabular}
		      }
		\end{table}
		\item Se aplican sobre los usuarios:
		\begin{itemize}
			\item Usuario: permisos del dueño $\rightarrow$ \textbf{U}
			\item Usuario: permisos del grupo $\rightarrow$ \textbf{G}
			\item Usuario: permisos de otros usuario $\rightarrow$ \textbf{O}
		\end{itemize}
		\item Se utiliza el comando \textbf{chmod}:
		\begin{lstlisting}
$ chmod 755 /tmp/script
		\end{lstlisting}
  \end{itemize}
\end{frame}
\end{comment}