\begin{frame}
  \frametitle{Definición Procesos}
  \begin{itemize}
	  \item Programa en ejecución
	  \item Los conceptos de tarea, job y proceso hacen referencia a lo mismo
	  \item Según su historial de ejecución, los podemos clasificar en:
	  \begin{itemize}
	  	\item CPU Bound (ligados a la CPU)
	  	\item I/O Bound (ligados a entrada/salida)
	  \end{itemize}
  \end{itemize}
\end{frame}

\begin{frame}
  \frametitle{Definición Procesos (cont.)}
  \begin{itemize}
	  \item Programa
	  \begin{itemize}
	  	\item Es estático
	  	\item No tiene program cunter
	  	\item Existe desde que se edita hasta que se borra
	  \end{itemize}
	  \begin{figure}
		    \includegraphics[scale=0.1]{images/process.png}
	  \end{figure}
	  \item Proceso
	  \begin{itemize}
	  	\item Es dinámico
	  	\item Tiene program counter
	  	\item Su ciclo de vida comprende desde que se lo ejecuta hasta que termina
	  \end{itemize}
	  \begin{figure}
			\includegraphics[scale=0.1]{images/program.png}
	  \end{figure}	  
  \end{itemize}
\end{frame}

\begin{frame}
  \frametitle{Procesos - PCB}
  \begin{itemize}
	  \item Una por proceso
	  \item Contiene información del proceso
	  \item Es lo primero que se crea cuando se realiza un \textit{fork} y lo último que se desaloca cuando termina
	  \begin{figure}
			\includegraphics[scale=0.2]{images/pcb.png}
	  \end{figure}	  
  \end{itemize}
\end{frame}

\begin{frame}
  \frametitle{Procesos - Estados}
  \begin{figure}
		\includegraphics[scale=0.5]{images/statesProcess.png}
  \end{figure}
\end{frame}

\begin{frame}
  \frametitle{Planificadores}
  \begin{itemize}
	  \item Es la clave de la multiprogramación
	  \item Esta diseñado de manera apropiada para cumplir las metas de:
	  \begin{itemize}
	  	\item Menor Tiempo de Respuesta
	  	\item Mayor rendimiento
	  	\item Uso eficiente del procesador
	  \end{itemize}
  \end{itemize}
\end{frame}

\begin{frame}
  \frametitle{Planificadores - Tipos}
  \begin{itemize}
		\item \textit{Long term scheduler}: admite nuevos procesos a memoria (controla el grado de multirpogramación)
		\item \textit{Medium term scheduler}: realiza el \emph{swapping} (intercambio) entre el disco y la memoria cuando el SO lo determina (puede disminuir el grado de multiprogramación)
		\item \textit{Short term scheduler}: determina que proceso pasará a ejecutarse
  \end{itemize}
\end{frame}

\begin{frame}
  \frametitle{Planificadores y Estados}
  \begin{figure}
    \includegraphics[scale=0.4]{images/statesSchedulers.png}
  \end{figure}
\end{frame}

\begin{frame}
  \frametitle{Planificadores y Colas}
  \begin{figure}
    \includegraphics[scale=0.4]{images/queuesSchedulers.png}
  \end{figure}
\end{frame}

\begin{frame}
  \frametitle{Tiempos de los procesos}
  \begin{itemize}
		\item \textbf{Retorno}: tiempo que transcurre entre que el proceso llega al sistema hasta que completa su ejecución
		\item \textbf{Espera}: tiempo que el proceso se encuentra en el sistema esperando, es decir el tiempo que pasa sin ejecutarse (\textbf{TR - Tcpu})
		\item \textbf{Promedios}: tiempos promedio de los anteriores
  \end{itemize}
\end{frame}

\begin{frame}
  \frametitle{Apropiación vs No Apropiación}
  \begin{itemize}
		\item \textbf{Nonpreemptive}: una vez que un proceso esta en estado de ejecución, continua hasta que termina o se bloquea por algún evento (e.j. I/O)
		\item \textbf{Preemptive}: el proceso en ejecución puede ser interrumpido y llevado a la cola de listos:
		\begin{itemize}
			\item Mayor overhead pero mejor servicio
			\item Un proceso no monopoliza el procesador
		\end{itemize}
  \end{itemize}
\end{frame}

\begin{frame}
  \frametitle{Algoritmo \textbf{FIFO}}
  \begin{itemize}
		\item Firs come first served
		\item Cuando hay que elegir un proceso para ejecutar, se selecciona el mas viejo
		\item No favorece a ningún tipo de procesos, pero en principio prodíamos decir que los \textit{CPU Bound} terminan al comenzar su primer ráfaga, mientras que los \textit{I/O Bound} no
  \end{itemize}
\end{frame}

\begin{frame}[fragile]
  \frametitle{Algoritmo \textbf{FIFO} (cont.)}
  \begin{table}
      \centering
      \resizebox{15pc}{!}{
	  \begin{tabular}{| c | c | c | c |}
	      \hline
	      \bf Job & \bf Llegada & \bf CPU & \bf Prioridad \\
	      \hline
	      1 & 0 & 9 & 3 \\
	      \hline
	      2 & 1 & 5 & 2 \\
	      \hline
	      3 & 2 & 3 & 1 \\
	      \hline
	      4 & 3 & 7 & 2 \\
	      \hline
	  \end{tabular}
      }
  \end{table}  
  \begin{lstlisting}
#Ejemplo 1
TAREA ''1'' PRIORIDAD=3
[CPU,9]
TAREA ''2'' PRIORIDAD=2
[CPU,5]
TAREA ''3'' PRIORIDAD=1
[CPU,3]
TAREA ''4'' PRIORIDAD=2
[CPU,7]  
  \end{lstlisting}  
  \hspace{35pt} \textcolor{orange}{¿Cuáles serían los tiempos de retorno y espera?}
\end{frame}

\begin{frame}
  \frametitle{Algoritmo \textbf{SJF}}
  \begin{itemize}
  		\item Shortest Job First
		\item Política \textit{nonpreemptive} que selecciona el proceso con la ráfaga más corto
		\item Procesos cortos se colocan delante de procesos largos
		\item Los procesos largos pueden sufrir \textit{starvation} (inanición)
  		\pause
  		\item \textcolor{orange}{Veamos el ejemplo anterior}	
  \end{itemize}
\end{frame}

\begin{frame}
  \frametitle{Algoritmo \textbf{RR}}
  \begin{itemize}
  		\item Round Robin
		\item Politica basada en un reloj
		\item \textbf{Quantum (Q)}: medida que determina cuanto tiempo podrá usar el procesador cada preceso:
		\begin{itemize}
			\item Pequeño: overhead de \textit{context switch}
			\item Grande: ¿pensar?
		\end{itemize}
		\item Cuando un proceso es expulsado de la \textit{CPU} es colocado al final de la \textit{Ready Queue} y se selecciona otro (\textit{FIFO circular})
  \end{itemize}
\end{frame}

\begin{frame}
  \frametitle{Algoritmo \textbf{RR} (cont.)}
  \begin{itemize}
  		\item Existe un ``contador'' que indica las unidades de CPU en las que el proceso se ejecuto. Cuando el mismo llega a 0 el proceso es expulsado
		\item Existen dos variantes con respecto al valor inicial del ``contador'' cuando un proceso es asignado a la CPU:
		\begin{itemize}
			\item \textbf{Timer Variable}
			\item \textbf{Timer Fijo}
		\end{itemize}
  \end{itemize}
\end{frame}

\begin{frame}
  \frametitle{Algoritmo \textbf{RR - Timer Variable}}
  \begin{itemize}
  		\item El ``contador'' se inicializa en Q (contador := Q) cada vez que un proceso es asignado a la \emph{CPU}
		\item Es el más utilizado
		\item Utilizado por el simulador
		\pause
		\item \textcolor{orange}{Veamos el ejemplo 1 nuevmanete}
  \end{itemize}
\end{frame}

\begin{frame}
  \frametitle{Algoritmo \textbf{RR - Timer Fijo}}
  \begin{itemize}
  		\item El ``contador'' se inicializa en Q cuando su valor es cero
  		\begin{itemize}
  			\item if (contador == 0) contador = Q;
  		\end{itemize}
		\item Se puede ver como un valor de Q compartido entre los procesos
  \end{itemize}
\end{frame}

\begin{frame}
  \frametitle{Algoritmo \textbf{RR} (cont.)}
	\begin{figure}
	    \includegraphics[scale=0.4]{images/ejemploRR.png}
	\end{figure}
\end{frame}

\begin{frame}
  \frametitle{Algoritmo con \textbf{Uso de Prioridades}}
  \begin{itemize}
  		\item Cada proceso tiene un valor que representa su prioridad $\rightarrow$ menor valor, mayor prioridad
		\item Se selecciona el proceso de mayor prioridad de los que se encuentran ela \emph{Ready Queue}
		\item Existe una \emph{Ready Queue} por cada nivel de prioridad
		\item Procesos de baja prioridad pueden sufrir \emph{starvation} (inanición)
		\begin{itemize}
			\item Solución: permitir a un proceso cambiar su prioridad durante su ciclo de vida $\rightarrow$ \textbf{Aging}
		\end{itemize}
		\item Es un algoritmo \textbf{preemptive}
		\pause
		\item \textcolor{orange}{Veamos el ejemplo 1 nuevamente}
  \end{itemize}
\end{frame}

\begin{frame}
  \frametitle{Algoritmo con Uso de Prioridades (cont.)}
	\begin{figure}
	    \includegraphics[scale=0.4]{images/priorities.png}
	\end{figure}
\end{frame}

\begin{frame}
  \frametitle{Algoritmo \textbf{SRTF}}
  \begin{itemize}
  		\item Shortest Remaining Time First  		
		\item Versión \emph{preemptive} de \textit{SJF}
		\item Selecciona el proceso al cual le resta menos tiempo de ejecución en su siguiente ráfaga.
		\item ¿A qué tipos de procesos favorece?
		\pause
		\textbf{$\rightarrow$ I/O Bound}
		\pause
		\item \textcolor{orange}{Veamos el ejemplo 1 nuevamente}
  \end{itemize}
\end{frame}

\begin{frame}
  \frametitle{Algoritmos de planificación - CPU + I/O}
  \begin{itemize}
  		\item Ciclo de vida de un proceso: uso de CPU + operaciones de I/O
		\item Cada dispositivo tiene su cola de procesos en espera $\rightarrow$ un scheduler por cada cola
		\item Se considera I/O independiente de la CPU (DMA, PCI, etc.) $\rightarrow$ uso de CPU y operaciones de I/O en simultaneo
  \end{itemize}
\end{frame}

\begin{frame}
  \frametitle{Algoritmos de planificación - Criterios de desempate}
  \begin{itemize}
  		\item Orden de aplicación:
  		\begin{itemize}
  			\item Orden de llegada de los procesos
  			\item \textbf{PID} de los procesos
  		\end{itemize}

		\item Siempre se mantiene la misma politica
  \end{itemize}
\end{frame}

\begin{frame}[fragile]
  \frametitle{Algoritmos de planificación - Un recurso por proceso}
  \begin{table}
      \centering
      \resizebox{15pc}{!}{
		  \begin{tabular}{| c | c | c | c |}
		      \hline
		      \bf Job & \bf Llegada & \bf CPU & \bf E/S (rec., inst., dur.) \\
		      \hline
		      1 & 0 & 5 & (R1, 3, 2) \\
		      \hline
		      2 & 1 & 4 & (R2, 2, 2) \\
		      \hline
		      3 & 2 & 3 & (R3, 2, 3) \\
		      \hline
		  \end{tabular}
      }
  \end{table}
  \begin{lstlisting}
#Ejemplo 2
RECURSO ''R1''
RECURSO ''R2''
RECURSO ''R3''
TAREA ''1'' INICIO=0
[CPU,3] [1,2] [CPU,2]
TAREA ''2'' INICIO=1
[CPU,2] [2,2] [CPU,2]
TAREA ''3'' INICIO=2
[CPU,2] [3,3] [CPU,1]
  \end{lstlisting}
\end{frame}

\begin{frame}[fragile]
  \frametitle{Algoritmos de planificación - Recurso compartido}
  \begin{table}
      \centering
      \resizebox{15pc}{!}{
		  \begin{tabular}{| c | c | c | c |}
		      \hline
		      \bf Job & \bf Llegada & \bf CPU & \bf E/S (rec., inst., dur.) \\
		      \hline
		      1 & 0 & 5 & (R1, 3, 3) \\
		      \hline
		      2 & 1 & 4 & (R1, 1, 2) \\
		      \hline
		      3 & 2 & 3 & (R2, 2, 3) \\
		      \hline
		  \end{tabular}
      }
  \end{table}
  \begin{lstlisting}
#Ejemplo 3
RECURSO ''R1''
RECURSO ''R1''
TAREA ''1'' INICIO=0
[CPU,3] [1,3] [CPU,2]
TAREA ''2'' INICIO=1
[CPU,1] [1,2] [CPU,3]
TAREA ''3'' INICIO=2
[CPU,2] [2,3] [CPU,1]
  \end{lstlisting}
\end{frame}

\begin{frame}
  \frametitle{Esquema \textbf{Colas Multinivel}}
  \begin{itemize}
  		\item Scheduler actuales $\rightarrow$ combinación de algoritmos vistos
		\item La \emph{ready queue} es dividida en varias colas (similar a prioridades)		
		\item Los procesos se colocan en las colas según una clasificación que realise el sistema operativo
		\item Cada cola posee su propio algoritmo de planificación $\rightarrow$ \textbf{planificador horizontal}
		\item A su vez existe un algoritmo que planifica las colas $\rightarrow$ \textbf{planificador vertical}
		\item Retroalimentacion $\rightarrow$ un proceso puede cambiar de una cola a la otra
  \end{itemize}
\end{frame}

\begin{frame}
  \frametitle{Esquema \textbf{Colas Multinivel} (ejemplo 1)}
	\begin{figure}
	    \includegraphics[scale=0.4]{images/multilevelSchemaExample1.png}
	\end{figure}
\end{frame}

\begin{frame}
  \frametitle{Esquema \textbf{Colas Multinivel} (ejemplo 2)}
  \begin{itemize}
  		\item El sistema consta de tres colas:
  		\begin{itemize}
  			\item Q0: se planifica con \emph{RR, q=8}
  			\item Q1: se planifica con \emph{RR, q=16}
  			\item Q2: se planifica con \emph{FCFS}
  		\end{itemize}
  		\item Para la planificacion se utilizan los siguientes criterios:
  		\begin{itemize}
  			\item Los procesos ingresan en la \emph{Q0}. Si no se utilizan los 8 cuantos el job es movido a la cola \emph{Q1}
  			\item Para la cola \emph{Q1}, el comportamiento es similar a \emph{Q0}. Si un proceso no finaliza su ráfaga de 16 instantes, es movido a la cola \emph{Q2}
  		\end{itemize}
  \end{itemize}
\end{frame}

\begin{frame}
  \frametitle{Esquema \textbf{Colas Multinivel} (ejemplo 2 cont.)}
	\begin{figure}
    	\includegraphics[scale=0.4]{images/multilevelSchemaExample2.png}
	\end{figure}
	\begin{itemize}
		\pause
		\item \textcolor{orange}{¿A qué procesos beneficia el algoritmo?}
		\pause
		$\rightarrow$ \textbf{CPU Bound}
	\pause  			
		\item \textcolor{orange}{¿Puede ocurrir inanición?}
	\pause
		$\rightarrow$ Si, con los procesos ligados a \emph{E/S} si siempre llegan procesos ligados a \emph{CPU}
	\end{itemize}
\end{frame}

\begin{frame}
  \frametitle{Planificación con múltiples procesadores}
	\begin{itemize}
		\item La planificación de CPU es más compleja cuando hay múltiples CPUs	
		\item Este enfoque fue implementado inicialmente en \textit{Mainframes} y luego en \textit{PC}
		\item La carga se divide entre distintas CPUs, logrando capacidades de procesamiento mayores
		\item Si un procesador falla, el resto toma el control
		\item La asignación de procesos a un procesador puede ser:
		\begin{itemize}
			\item \emph{Estática}: existe una afinidad de un proceso a una CPU
			\item \emph{Dinámica}: la carga se comparte
		\end{itemize}
	\end{itemize}
\end{frame}

\begin{frame}
  \frametitle{Planificación con múltiples procesadores (cont.)}
	\begin{itemize}		
		\item \emph{Clasificaciones}:
		\begin{itemize}
			\item \textbf{Procesadores homogéneos}: todas las CPUs son iguales. No existen ventajas físicas sobre el resto
			\item \textbf{Procesadores heterogéneos}: cada procesador tiene su propia cola y algoritmo de planificación
		\end{itemize}
		\item \emph{Otra clasificación}:
		\begin{itemize}
			\item \textbf{Procesadores débilmente acoplados}: cada CPU tiene su propia memoria principal y canales
			\item \textbf{Procesadores fuertenemente acoplados}:comparten memoria y canales
			\item \textbf{Procesadores especializados}: uno o más procesadores principales de uso general y uno o más procesadores de uso específico
		\end{itemize}		
	\end{itemize}
\end{frame}