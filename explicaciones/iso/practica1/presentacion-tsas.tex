\begin{frame}
  \frametitle{La Cátedra}
  \begin{itemize}
	  \item Sitio Web: \textcolor{orange}{\url{http://catedras.info.unlp.edu.ar}} (único medio de comunicación)
	  \item Explicaciones:
	  \begin{itemize}
	  	\item TM: sábados 9 hs.
	  \end{itemize}
	  \item Integrantes:
	  \begin{itemize}
	  	\item Profesor: Carlos Meza
	  	\item JTP: Máximo Zarza $\rightarrow$ \textcolor{orange}{zmaximo1990@gmail.com}
	  \end{itemize}
  \end{itemize}
\end{frame}

\begin{frame}
  \frametitle{Cursada}
  \begin{itemize}
	  \item La evaluación de la práctica de la materia constara de seis (6) parciales, uno por cada trabajo práctico de la cursada.
	  \item Para aprobar la cursada, el alumno deberá tener aprobado cinco (5) de los seis (6) parciales
  \end{itemize}
\end{frame}

\begin{frame}
  \frametitle{Cursada}
  \begin{itemize}
	  \item Finalizada la cursada, el alumno que no haya aprobado cinco (5) parciales y que:
	  \begin{itemize}
	  	\item Haya rendido al menos cuatro (4) de los seis (6) parciales
	  	\item Y haya aprobado al menos dos (2) de los seis (6)
	  \end{itemize}
	  
	  \pause
	  \textcolor{orange}{La cátedra dispondrá de una ÚNICA fecha en la que el alumno podrá recuperar los temas adeudados.}
  \end{itemize}
\end{frame}

\begin{frame}
  \frametitle{Los parciales}
  \begin{itemize}
	  \item Entra todo lo visto en la práctica (enunciados de TP, explicaciones, material adicional brindado por la cátedra, respuestas en el foro)
	  \item Las fechas de los parciales prácticos serán en la clase próxima al trabajo práctico y se evaluara al comienzo de la clase correspondiente. Todo cambio de fecha que pudiera ocurrir se comunicara con la debida antelación por medio de la plataforma web
	  \item El alumno debe encontrarse en la cursada de ISO \the\year\ para poder rendir los parciales. Si, llegado el quinto (5to) parcial y el alumno no se encuentrara inscripto correctamente en el sistema SIU-GUARANI, el mismo NO podrá rendir y automáticamente pierde los parciales ya rendidos.
  \end{itemize}
\end{frame}

\begin{frame}
	\frametitle{Cronograma}
	\begin{table}
	      \centering
	      \resizebox{20pc}{!}{
		  \begin{tabular}{|c|c|}
		  		\hline
		  		\bf Fecha & \bf Práctica/Tarea \\
		  		\hline
		  		\bf \textcolor{orange}{12/09} & Práctica 1: Conceptos generales \\
		  		\hline
		  		\bf \textcolor{orange}{26/09} & Parcialito 1 + Práctica 2: Introducción a GNU/Linux \\
		  		\hline
		  		\bf \textcolor{orange}{10/10} & Parcialito 2 + Práctica 3: Shell scripting \\
		  		\hline
		  		\bf \textcolor{orange}{24/10} & Parcialito 3 + Práctica 4: Administracción de procesos \\
		  		\hline
		  		\bf \textcolor{orange}{07/11} & Parcialito 4 + Práctica 5: Administración de memoria \\
		  		\hline
		  		\bf \textcolor{orange}{21/11} & Parcialito 5 + Práctica 6: Administración de E/S \\
		  		\hline
		  		\bf \textcolor{orange}{05/12} & Parcial general \\
		  		\hline
		  \end{tabular}
	      }
	\end{table}
\end{frame}
